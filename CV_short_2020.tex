%%%%%%%%%%%%%%%%%%%%%%%%%%%%%%%%%%%%%%%%%%%%%%%%%%%%%%%%%%%%%%%%%%%%%%%%
%%%%%%%%%%%%%%%%%%%%%% Simple LaTeX CV Template %%%%%%%%%%%%%%%%%%%%%%%%
%%%%%%%%%%%%%%%%%%%%%%%%%%%%%%%%%%%%%%%%%%%%%%%%%%%%%%%%%%%%%%%%%%%%%%%%

%%%%%%%%%%%%%%%%%%%%%%%%%%%%%%%%%%%%%%%%%%%%%%%%%%%%%%%%%%%%%%%%%%%%%%%%
%% NOTE: If you find that it says                                     %%
%%                                                                    %%
%%                           1 of ??                                  %%
%%                                                                    %%
%% at the bottom of your first page, this means that the AUX file     %%
%% was not available when you ran LaTeX on this source. Simply RERUN  %%
%% LaTeX to get the ``??'' replaced with the number of the last page  %%
%% of the document. The AUX file will be generated on the first run   %%
%% of LaTeX and used on the second run to fill in all of the          %%
%% references.                                                        %%
%%%%%%%%%%%%%%%%%%%%%%%%%%%%%%%%%%%%%%%%%%%%%%%%%%%%%%%%%%%%%%%%%%%%%%%%

%%%%%%%%%%%%%%%%%%%%%%%%%%%% Document Setup %%%%%%%%%%%%%%%%%%%%%%%%%%%%

% Don't like 10pt? Try 11pt or 12pt
\documentclass[10pt]{article}

% This is a helpful package that puts math inside length specifications
\usepackage{calc,hyperref}
\usepackage[utf8]{inputenc}
\usepackage{amsmath,amssymb}



% Simpler bibsection for CV sections
% (thanks to natbib for inspiration)
\makeatletter
\newlength{\bibhang}
\setlength{\bibhang}{1em}
\newlength{\bibsep}
 {\@listi \global\bibsep\itemsep \global\advance\bibsep by\parsep}
\newenvironment{bibsection}
    {\minipage[t]{\linewidth}\list{}{%
        \setlength{\leftmargin}{\bibhang}%
        \setlength{\itemindent}{-\leftmargin}%
        \setlength{\itemsep}{\bibsep}%
        \setlength{\parsep}{\z@}%
        }}
    {\endlist\endminipage}
\makeatother

% Layout: Puts the section titles on left side of page
\reversemarginpar

%
%         PAPER SIZE, PAGE NUMBER, AND DOCUMENT LAYOUT NOTES:
%
% The next \usepackage line changes the layout for CV style section
% headings as marginal notes. It also sets up the paper size as either
% letter or A4. By default, letter was used. If A4 paper is desired,
% comment out the letterpaper lines and uncomment the a4paper lines.
%
% As you can see, the margin widths and section title widths can be
% easily adjusted.
%
% ALSO: Notice that the includefoot option can be commented OUT in order
% to put the PAGE NUMBER *IN* the bottom margin. This will make the
% effective text area larger.
%
% IF YOU WISH TO REMOVE THE ``of LASTPAGE'' next to each page number,
% see the note about the +LP and -LP lines below. Comment out the +LP
% and uncomment the -LP.
%
% IF YOU WISH TO REMOVE PAGE NUMBERS, be sure that the includefoot line
% is uncommented and ALSO uncomment the \pagestyle{empty} a few lines
% below.
%

%% Use these lines for letter-sized paper
\usepackage[paper=letterpaper,
            %includefoot, % Uncomment to put page number above margin
            marginparwidth=.9in,     % Length of section titles
            marginparsep=.05in,       % Space between titles and text
            margin=.7in,               % 1 inch margins
            includemp]{geometry}

%% Use these lines for A4-sized paper
%\usepackage[paper=a4paper,
%            %includefoot, % Uncomment to put page number above margin
%            marginparwidth=30.5mm,    % Length of section titles
%            marginparsep=1.5mm,       % Space between titles and text
%            margin=25mm,              % 25mm margins
%            includemp]{geometry}

%% More layout: Get rid of indenting throughout entire document
\setlength{\parindent}{0in}

%% This gives us fun enumeration environments. compactitem will be nice.
\usepackage{paralist}

%% Reference the last page in the page number
%
% NOTE: comment the +LP line and uncomment the -LP line to have page
%       numbers without the ``of ##'' last page reference)
%
% NOTE: uncomment the \pagestyle{empty} line to get rid of all page
%       numbers (make sure includefoot is commented out above)
%
\usepackage{fancyhdr,lastpage}
\pagestyle{fancy}
%\pagestyle{empty}      % Uncomment this to get rid of page numbers
\fancyhf{}\renewcommand{\headrulewidth}{0pt}
\fancyfootoffset{\marginparsep+\marginparwidth}
\newlength{\footpageshift}
\setlength{\footpageshift}
          {0.5\textwidth+0.5\marginparsep+0.5\marginparwidth-2in}
\lfoot{\hspace{\footpageshift}%
       \parbox{4in}{\, \hfill %
                    \arabic{page} of \protect\pageref*{LastPage} % +LP
%                    \arabic{page}                               % -LP
                    \hfill \,}}

% Finally, give us PDF bookmarks
\usepackage{color,hyperref}
\definecolor{darkblue}{rgb}{0.0,0.0,0.3}
\hypersetup{colorlinks,breaklinks,
            linkcolor=darkblue,urlcolor=darkblue,
            anchorcolor=darkblue,citecolor=darkblue}

%%%%%%%%%%%%%%%%%%%%%%%% End Document Setup %%%%%%%%%%%%%%%%%%%%%%%%%%%%


%%%%%%%%%%%%%%%%%%%%%%%%%%% Helper Commands %%%%%%%%%%%%%%%%%%%%%%%%%%%%

% The title (name) with a horizontal rule under it
%
% Usage: \makeheading{name}
%
% Place at top of document. It should be the first thing.
\newcommand{\makeheading}[1]%
        {\hspace*{-\marginparsep minus \marginparwidth}%
         \begin{minipage}[t]{\textwidth+\marginparwidth+\marginparsep}%
                {\large \bfseries #1}\\[-0.15\baselineskip]%
                 \rule{\columnwidth}{1pt}%
         \end{minipage}}

% The section headings
%
% Usage: \section{section name}
%
% Follow this section IMMEDIATELY with the first line of the section
% text. Do not put whitespace in between. That is, do this:
%
%       \section{My Information}
%       Here is my information.
%
% and NOT this:
%
%       \section{My Information}
%
%       Here is my information.
%
% Otherwise the top of the section header will not line up with the top
% of the section. Of course, using a single comment character (%) on
% empty lines allows for the function of the first example with the
% readability of the second example.
\renewcommand{\section}[2]%
        {\pagebreak[2]\vspace{.2\baselineskip}%
         \phantomsection\addcontentsline{toc}{section}{#1}%
         \hspace{0in}%
         \marginpar{
         \raggedright \scshape #1}#2}

% An itemize-style list with lots of space between items
\newenvironment{outerlist}[1][\enskip\textbullet]%
        {\begin{itemize}[#1]}{\end{itemize}%
         \vspace{-.6\baselineskip}}

% An environment IDENTICAL to outerlist that has better pre-list spacing
% when used as the first thing in a \section
\newenvironment{lonelist}[1][\enskip\textbullet]%
        {\vspace{-\baselineskip}\begin{list}{#1}{%
        \setlength{\partopsep}{0pt}%
        \setlength{\topsep}{0pt}}}
        {\end{list}\vspace{-.6\baselineskip}}

% An itemize-style list with little space between items
\newenvironment{innerlist}[1][\enskip\textbullet]%
        {\begin{compactitem}[#1]}{\end{compactitem}}

% To add some paragraph space between lines.
% This also tells LaTeX to preferably break a page on one of these gaps
% if there is a needed pagebreak nearby.
\newcommand{\blankline}{\quad\pagebreak[2]}

%--- Ignore what follows
\newcommand{\Ignore}[1]{}

% 

%%%%%%%%%%%%%%%%%%%%%%%% End Helper Commands %%%%%%%%%%%%%%%%%%%%%%%%%%%

%%%%%%%%%%%%%%%%%%%%%%%%% Begin CV Document %%%%%%%%%%%%%%%%%%%%%%%%%%%%

\begin{document}
\makeheading{Llohann Dallagnol Sperança}

\section{}
%
% NOTE: Mind where the & separators and \\ breaks are in the following
%       table.
%
% ALSO: \rcollength is the width of the right column of the table
%       (adjust it to your liking; default is 1.85in).
%

%--- Move two lines up then add contact info; this is a bit of a hack; also might want to move up a bit more
\vspace{-1em}
%\begin{tabbing}
% Federal University of São Paulo xxxxxxxxxxxxxxl \= \kill
%%\href{http://www.ime.unicamp.br/} 
%\\
%{Federal University of São Paulo}    \> \textit{Cell:} +55 12 9 9788 7860\\
%%\href{http://www.unicamp.br/unicamp/}
%{Department of Science and Technology}
%                           \>  \textit{E-mail:} {speranca@unifesp.br}
%%Av. Cesare M. G. Lattes, 1201          \>  \\
%%São José dos Campos  – São Paulo          \>  \textit{Nationality:} Brazilian, Italian \\
%%12247-014, Brazil 
%\end{tabbing}

%		\rule{\columnwidth}{.5pt}%


%\section{Citzenship}
%%
%Brazilian


%\section{Research Interests}
%%
%My main interest is in Riemannian geometry, specially in its interplay between topology and curvature expressed in Riemannian/metric foliations. I also have been working in other areas, including geometric analysis, mathematical physics and combinatorics.
%
%		\rule{\columnwidth}{.5pt}%


\section{Education}
%
\begin{innerlist}
	
\item[] Ph.D.,
        \href{http://www.ime.unicamp.br/}
             {Mathematics}
            May 2012, \hfill State University of Campinas

%\blankline

\item[] M.S.,
        \href{http://www.ime.unicamp.br/}
             {Mathematics}, August 2009, \hfill 
             State University of Campinas
%        \begin{innerlist}
%	\item Thesis Topic: \emph{Exotic Phenomena in Geometry and Topology}
%	\item Adviser:
%	\href{http://lattes.cnpq.br/2287909020835559}
%	{Professor Carlos Durán} 
%	\item Area of Study: Differential Topology, Riemannian Geometry
%\end{innerlist}

%\blankline

\item[] B.S.,
        \href{http://www.ime.unicamp.br/}
             {Mathematics}, December 2007, \hfill State University of Campinas
\end{innerlist}

		\rule{\columnwidth}{.5pt}%


\section{Positions}\vspace{-1em}
%	\item[]

	 Assistent professor at {Federal University of São Paulo},  \hfill Since Spring 2017 

Visiting Researcher at Universit\"at Zu K\"oln, \hfill Mar 2018 to Feb 2019
	
	
%	\blankline
	
	Assistent professor  at {Federal University of Paraná}, \hfill Spring 2014 to Spring 2017

%\blankline

Postdoctoral fellow at {State University of Campinas} \hfill Spring 2013 to Spring 2014
\rule{\columnwidth}{.5pt}%

%\section{Honors and Awards}

%FAPESP Grant for Research Abroad: Classification and global properties of Riemannian foliations, 2017/19657-0, \$42,000, \hfill 2018-2019


%Bronze Medal for the work `Geometric Aspects in Gauge Theory' -- III Journey  on Scientific Initiation, IMPA, Rio de Janeiro, \hfill 2006



%\rule{\columnwidth}{.5pt}%
%
%
%\section{Grants}
% Event Support: Advanced Program in Geometry -- Federal University of Paraná, CAPES 7197/2014-90, CAPES 7891/2014-15, CNPq 466425/2014-7,  PCDP  000076/15, \$4,000 \hfill Summer\,\,2015
%
%\blankline
%
%CNPq Universal Call: Riemannian Foliations under positive/nonnegative seccional curvature, 404266/2016-9 , \$4,000, \hfill 2016-2020
%
%\blankline
%
% FAPESP Research Grant: Geometry and topology under positive/nonnegative sectional curvature, 2017/10892-7, \$3,100, \hfill 2017-2018
%
%\blankline
%
%
%
%
%\rule{\columnwidth}{.5pt}%

%\section{Shor-term positions} 
%\begin{innerlist}
%\item[2012-2013] Postdoctoral Researcher for the state of São Paulo, Brazil at the State University of Campinas
%\item[2018-2019] Visiting Professor at the University of K\"oln, Germany
%\end{innerlist}
%
%\rule{\columnwidth}{.5pt}%


\section{Selected Papers} \vspace{-1em}
\begin{enumerate}
%	\item Jardim, M. B., Sperança, L. D. Nonsingular Complex Instantons on Euclidean Spacetime. \textit{International Journal of Geometric Methods in Modern Physics}, v. 05, p. 963, 2008.
%	%    \vspace{.1 in}      
%	
%	\item Durán, C. E. , A. Rigas, A., Sperança, L. D.  Bootstrapping Ad-equivariant maps, Diffeomorphims and Involutions. \textit{Matematica Contemporanea},  v. 35, p. 27-39, 2008.
%	%    \vspace{.1 in}
%	
%	\item Sperança L. D. A note on the degree of symmetry of exotic spheres. \textit{Archiv der Mathematik (Printed ed.)},  v. 97, p. 495-497, 2011.
%	%    \vspace{.1 in}  	  
%	
%	\item Sperança, L. D. An identification of the Dirac operator with the parity operator. \textit{International Journal of Modern Physics D},  v. 97, p. 495-497, 2011.
%	%    \vspace{.1 in}
%	
%	\item Durán, C. E., Sperança, L. D. Rigidity of flat sections on non-negatively curved pullback submersions, \textit{Manuscripta Mathematica}, v. 147, p. 511-525, 2015.
%	%    \vspace{.1 in}
%	
%	\item Sperança, L. D. Pulling back the Gromoll-Meyer construction and models of exotic spheres. \textit{Proceedings of the American Mathematical Society}, v. 144, p. 3181-3196, 2016.
	%    \vspace{.1 in}
	\item Sperança, L. D. On Riemannian foliations over positively curved manifolds. \textit{The Journal of Geometric Analysis}, v. 28(3), p. 2206-2224, 2017.
	%	\vspace{.1 in}  
	
	\item Sperança, L. D., Weil, S.  The metric foliations on Euclidean spaces, \textit{Mathematische Zeitschrift}, \href{https://link.springer.com/article/10.1007/s00209-019-02425-3}{Math. Z. 295, 1295–1299 2020.} 
	
	\item  Cavenaghi, L. F., Sperança, L. D. On the Geometry of Some Equivariantly Related Manifolds,  
\textit{ \href{https://academic.oup.com/imrn/advance-article-abstract/doi/10.1093/imrn/rny268/5194089}{International Mathematics Research
		Notices, no. 23, 9730–9768, 2020}}
	
		\item Silva, R. M., Sperança, L. D. On the completeness of dual foliations on nonnegatively curved symmetric spaces, \textit{\href{https://arxiv.org/abs/2006.13809}{arxiv:2006.13809}} accepted Revista Matemática Iberoamericana.
		
	\item Sperança, L. D. Totally Geodesic Riemannian Foliations on Compact Lie Groups, \textit{\href{https://arxiv.org/abs/1703.09577}{arxiv:1703.09577}}, submitted to The American Journal of Mathematics.
	%	\vspace{.1 in}
	
%	\item Sperança, L. D. An intrinsic curvature condition for submersions over Riemannian manifolds
%	\textit{\href{https://arxiv.org/abs/1706.09211}{arxiv:1706.09211}}.
%	
%	\item Cavenaghi, L. F., Sperança, L. D. A metric deformation on fiber bundles and applications, \textit{\href{https://arxiv.org/abs/1801.06576}{arxiv:1801.06576}}, submitted to Advances in Geometry.
	%	\vspace{.1 in}
	
%	\item Cavenaghi, L. F., Silva, R. M., Sperança, L. D. Positive Ricci curvature through Cheeger deformation,
%	\textit{\href{https://arxiv.org/abs/1810.09725}{arxiv:1810.09725}}, submitted The Journal of Geometric Analysis.
	%	\vspace{.1 in}
	
%	\item Grama, L., Martins, R. M., Patrão, M., Seco, L., Sperança, L. D.
%	Global dynamics of the Ricci flow on flag manifolds with three isotropy summands, \textit{\href{https://arxiv.org/abs/2004.01511}{arxiv:2004.01511}}, submitted to Revista Matemática Iberoamericana.
	

\end{enumerate}

\rule{\columnwidth}{.5pt}%


%\section{Book} 
%\begin{enumerate}
%	\item  Sperança, L. D. Grupos de Lie via exemplos: Topologia, Geometria e Física. ed. Rio de Janeiro SBM 2016 v.1 61p.
%\end{enumerate}
%
%
%
%\rule{\columnwidth}{.5pt}%


\section{Talks}\vspace{-1em}
%
%\textbf{``Positive Ricci curvature through Cheeger deformation" }
%\begin{innerlist}
%\item[] XII Summer School, Dynamical Systems section,  University of Brasília (Brazil), 2020
%\end{innerlist}

%\blankline

\textbf{``The metric foliations on Euclidean spaces"}
\begin{innerlist}
	\item[]- SP Geometry Seminar, State University of Campinas (Brazil), 2020
\end{innerlist}




\blankline

\textbf{``On the geometry of some equivariantly related manifolds"}
\begin{innerlist}
%\item[]- Dynamical Systems Seminar, University of Brasília (Brazil), 2020
\item[]- Oberseminar Topologie, University of Fribourg (Switzerland), 2019
\item[]- Obeseminar Geometrie, University of Freiburg (Germany), 2019
\item[]- Geometry Seminar, KU Leuven (Belguim), 2019
\end{innerlist}



\blankline

\textbf{``Totally Geodesic Riemannian Foliations on Compact Lie Groups"}
\begin{innerlist}
\item[]- AMS Sectional Meeting, University of California at Riverside (USA), 2017
\item[]- Workshop on Curvature and Global Shape, University of M\"unster (Germany), 2017
\item[]- Geometry Seminar, IMPA (Brazil), 2017
\item[]- Symposium on Lie Theory and Applications, State University of Maringá (Brazil), 2016
%\item[]- Geometry Seminar, State University of Campinas (Brazil), 2015
\end{innerlist}


\blankline

\textbf{``Riemannian foliations on positively curved manifolds"}
\begin{innerlist}
\item[]- Geometry Seminar, University of K\"oln (Germany), 2019
\item[]- Oberseminar Geometrie, University of M\"unster (Germany), 2015
%\item[]- Geometry Seminar, University of São Paulo (Brazil), 2015
%\item[]- Geometry Seminar, University of Santa Catarina (Brazil), 2015
\end{innerlist}


%\blankline
%
%\textbf{``Lorentz Algebras, Space-time symmetries and Dark-Matter''}
%\begin{innerlist}
%	\item[]- II School  and Workshop in Lie Theory, State University of Maringa (Brazil), 2013
%\end{innerlist}
%
%
%\blankline
%
%\textbf{``Geometric Aspects of Gauge Theory''}
%\begin{innerlist}
%	\item[]- III Journey on Scientific Initiation, IMPA (Brazil), 2006
%\end{innerlist}



\rule{\columnwidth}{.5pt}%


%
%\section{Academic Experience (as Instructor)}
%\textbf{\href{http://ime.unicamp.br}{State University of Campinas}}
%	\hfill {Summer 2013}
%	\begin{innerlist}
%            \item[]- Advanced Linear Algebra (graduation)\hfill  Summer 2013
%	\end{innerlist}
%
%\blankline
%
%\textbf{\href{http://www.mat.ufpr.br/}{Federal University of Paraná}}
%\hfill { 2014/02 to 2017/02}
%		\begin{innerlist}
%		\item[]- Calculus \hfill Spring 2014, Fall 2015 
%		\item[]- Calculus in multiple variables \hfill Spring 2014, Spring 2016, Fall 2016 
%		\item[]- Linear Algebra \hfill Spring 2015, Fall 2015, Spring 2016 
%		\item[]- Differential Geometry (Math major) \hfill Spring 2014
%		\item[]- Algebraic Topology (Math major) \hfill Fall 2014
%		\item[]- Topics in Geometry/Topology (Math Major) \hfill  Spring 2016
%		\item[]- Advanced Linear Algebra (graduation) \hfill Spring 2015 
%		\item[]- Ordinary Differential Equations (graduation) \hfill  Fall 2016 		
%		\item[]- Geometry and Topology (PhD course on manifolds) \hfill Fall 2014
%		
%	\end{innerlist}
%
%\blankline
%
%\textbf{\bfseries\href{https://www.unifesp.br/campus/sjc/}{Federal University of São Paulo}}
%\hfill {2017/02 to Present}
%		\begin{innerlist}
%	\item[]- Ordinary Differential Equations \hfill Fall 2019 
%	\item[]- Calculus in multiple variables\hfill  Spring 2019 
%	\item[]- Calculus\hfill  Spring 2018, Spring 2020 (interrupted due to the pandemic)
%	\item[]- Metric Spaces (Math major) \hfill Spring 2017 
%	\item[]- Ordinary Differential Equations (Math major) \hfill Fall 2017
%	\item[]- General Topology (graduation) \hfill Fall 2017 
%	\item[]- Advanced Linear Algebra (graduation)\hfill Summer 2018 
%	\item[]- Analysis on $\mathbb{R}^n$ II (graduation) \hfill Spring 2019 \item[]- Analysis on $\mathbb R^n$ (graduation) \hfill Fall 2019
%%	\item[]- Differential Geometry (Math major): Spring 2014
%%	\item[]- Algebraic Topology (Math major):Fall 2014
%%	\item[]- Geometry and Topology (PhD course on manifolds): Fall 2014
%%	\item[]- Linear Algebra: Spring 2015 (Mechanical Engineering), Fall 2015 (Production Engineering), Spring 2016 (Chemical Engineering)
%%	\item[]- Topics in Geometry/Topology (Introduction to Riemannian Geometry): Spring 2016
%\end{innerlist}
    
%\rule{\columnwidth}{.5pt}%
    
  \section{Students} 
  \textbf{PhD Thesis}
  	\begin{innerlist}
  	\item[2018--2020] Leonardo F. Cavenaghi: On metric deformations and applications, University of São Paulo
  	\item[2018--2021] Renato J. Moura e Silva: Results on Riemannian foliations, State University of Campinas
  	\end{innerlist}
  
  \blankline
%  
\hspace{-.5cm} \textbf{Masters Dissertation}
  \begin{innerlist}
  	\item[2013--2014] Aline Zanardini: Dirac Structures and their homotopy classification, Federal University of Paraná
  \end{innerlist}
%  
%  \blankline
%  
%  \textbf{Undergraduate Students}
%  \begin{innerlist}
%  	\item[2017]
%  	Hilário Fernandes de Araújo Júnior: Homology and Homotopy in Geometric Topology, Federal University of São Paulo
%  	\item[2016] Felipe Ogima: Introduction to Ordinary Differential Equations, Federal University of Paraná
%  	\item[2016] Luciano Luzzi Junior: Introduction to Fiber Bundles, Federal University of Paraná
%  	\item[2016] Gabriel José Goulart Cardoso: Topology and Geometry in Geometric Quantization, Federal University of Paraná
%  \end{innerlist}
%
%
%\rule{\columnwidth}{.5pt}%
%
%
%\section{Academic and\\ Professional Service}
%
%%\subsection{Conferences and Workshops} 
%%
%
%   	
%\textbf{Organization of events}
%   \begin{innerlist}
%   \item[]- Advanced Summer Program in Geometry, Federal University of Paraná, 2015
%   \item[]- III Geometry Symposium, Federal University of Paraná, 2015
%   \item[]- V Geometry Symposium, Federal University of Paraná, 2017
%   \end{innerlist}
%   
%   \blankline
%   \textbf{Reviewer Service}
%   \begin{innerlist}   
%\item[]- Project Reviewer for The São Paulo Research Foundation (Fapesp)
%   \end{innerlist}
%	 
%	 \blankline
%	 
%	 \textbf{Referee Service}
%		\begin{innerlist}
%   \item[]- Advances in Applied Clifford Algebras
%\item[]- International Journal of Geometric Methods in Modern Physics
%\item[]- SIGMA
%\item[]- Mediterranean Journal of Mathematics
%		\end{innerlist}
				

%\end{outerlist}




\end{document}

%%%%%%%%%%%%%%%%%%%%%%%%%% End CV Document %%%%%%%%%%%%%%%%%%%%%%%%%%%%%
