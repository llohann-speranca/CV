%%%%%%%%%%%%%%%%%%%%%%%%%%%%%%%%%%%%%%%%%%%%%%%%%%%%%%%%%%%%%%%%%%%%%%%%
%%%%%%%%%%%%%%%%%%%%%% Simple LaTeX CV Template %%%%%%%%%%%%%%%%%%%%%%%%
%%%%%%%%%%%%%%%%%%%%%%%%%%%%%%%%%%%%%%%%%%%%%%%%%%%%%%%%%%%%%%%%%%%%%%%%

%%%%%%%%%%%%%%%%%%%%%%%%%%%%%%%%%%%%%%%%%%%%%%%%%%%%%%%%%%%%%%%%%%%%%%%%
%% NOTE: If you find that it says                                     %%
%%                                                                    %%
%%                           1 of ??                                  %%
%%                                                                    %%
%% at the bottom of your first page, this means that the AUX file     %%
%% was not available when you ran LaTeX on this source. Simply RERUN  %%
%% LaTeX to get the ``??'' replaced with the number of the last page  %%
%% of the document. The AUX file will be generated on the first run   %%
%% of LaTeX and used on the second run to fill in all of the          %%
%% references.                                                        %%
%%%%%%%%%%%%%%%%%%%%%%%%%%%%%%%%%%%%%%%%%%%%%%%%%%%%%%%%%%%%%%%%%%%%%%%%

%%%%%%%%%%%%%%%%%%%%%%%%%%%% Document Setup %%%%%%%%%%%%%%%%%%%%%%%%%%%%

% Don't like 10pt? Try 11pt or 12pt
\documentclass[10pt]{article}

% This is a helpful package that puts math inside length specifications
\usepackage{calc,hyperref}
\usepackage[utf8]{inputenc}
\usepackage{amsmath,amssymb}



% Simpler bibsection for CV sections
% (thanks to natbib for inspiration)
\makeatletter
\newlength{\bibhang}
\setlength{\bibhang}{1em}
\newlength{\bibsep}
 {\@listi \global\bibsep\itemsep \global\advance\bibsep by\parsep}
\newenvironment{bibsection}
    {\minipage[t]{\linewidth}\list{}{%
        \setlength{\leftmargin}{\bibhang}%
        \setlength{\itemindent}{-\leftmargin}%
        \setlength{\itemsep}{\bibsep}%
        \setlength{\parsep}{\z@}%
        }}
    {\endlist\endminipage}
\makeatother

% Layout: Puts the section titles on left side of page
\reversemarginpar

%
%         PAPER SIZE, PAGE NUMBER, AND DOCUMENT LAYOUT NOTES:
%
% The next \usepackage line changes the layout for CV style section
% headings as marginal notes. It also sets up the paper size as either
% letter or A4. By default, letter was used. If A4 paper is desired,
% comment out the letterpaper lines and uncomment the a4paper lines.
%
% As you can see, the margin widths and section title widths can be
% easily adjusted.
%
% ALSO: Notice that the includefoot option can be commented OUT in order
% to put the PAGE NUMBER *IN* the bottom margin. This will make the
% effective text area larger.
%
% IF YOU WISH TO REMOVE THE ``of LASTPAGE'' next to each page number,
% see the note about the +LP and -LP lines below. Comment out the +LP
% and uncomment the -LP.
%
% IF YOU WISH TO REMOVE PAGE NUMBERS, be sure that the includefoot line
% is uncommented and ALSO uncomment the \pagestyle{empty} a few lines
% below.
%

%% Use these lines for letter-sized paper
\usepackage[paper=letterpaper,
            %includefoot, % Uncomment to put page number above margin
            marginparwidth=.9in,     % Length of section titles
            marginparsep=.05in,       % Space between titles and text
            margin=.7in,               % 1 inch margins
            includemp]{geometry}

%% Use these lines for A4-sized paper
%\usepackage[paper=a4paper,
%            %includefoot, % Uncomment to put page number above margin
%            marginparwidth=30.5mm,    % Length of section titles
%            marginparsep=1.5mm,       % Space between titles and text
%            margin=25mm,              % 25mm margins
%            includemp]{geometry}

%% More layout: Get rid of indenting throughout entire document
\setlength{\parindent}{0in}

%% This gives us fun enumeration environments. compactitem will be nice.
\usepackage{paralist}

%% Reference the last page in the page number
%
% NOTE: comment the +LP line and uncomment the -LP line to have page
%       numbers without the ``of ##'' last page reference)
%
% NOTE: uncomment the \pagestyle{empty} line to get rid of all page
%       numbers (make sure includefoot is commented out above)
%
\usepackage{fancyhdr,lastpage}
\pagestyle{fancy}
%\pagestyle{empty}      % Uncomment this to get rid of page numbers
\fancyhf{}\renewcommand{\headrulewidth}{0pt}
\fancyfootoffset{\marginparsep+\marginparwidth}
\newlength{\footpageshift}
\setlength{\footpageshift}
          {0.5\textwidth+0.5\marginparsep+0.5\marginparwidth-2in}
\lfoot{\hspace{\footpageshift}%
       \parbox{4in}{\, \hfill %
                    \arabic{page} of \protect\pageref*{LastPage} % +LP
%                    \arabic{page}                               % -LP
                    \hfill \,}}

% Finally, give us PDF bookmarks
\usepackage{color,hyperref}
\definecolor{darkblue}{rgb}{0.0,0.0,0.3}
\hypersetup{colorlinks,breaklinks,
            linkcolor=darkblue,urlcolor=darkblue,
            anchorcolor=darkblue,citecolor=darkblue}

%%%%%%%%%%%%%%%%%%%%%%%% End Document Setup %%%%%%%%%%%%%%%%%%%%%%%%%%%%


%%%%%%%%%%%%%%%%%%%%%%%%%%% Helper Commands %%%%%%%%%%%%%%%%%%%%%%%%%%%%

% The title (name) with a horizontal rule under it
%
% Usage: \makeheading{name}
%
% Place at top of document. It should be the first thing.
\newcommand{\makeheading}[1]%
        {\hspace*{-\marginparsep minus \marginparwidth}%
         \begin{minipage}[t]{\textwidth+\marginparwidth+\marginparsep}%
                {\large \bfseries #1}\\[-0.15\baselineskip]%
                 \rule{\columnwidth}{1pt}%
         \end{minipage}}

% The section headings
%
% Usage: \section{section name}
%
% Follow this section IMMEDIATELY with the first line of the section
% text. Do not put whitespace in between. That is, do this:
%
%       \section{My Information}
%       Here is my information.
%
% and NOT this:
%
%       \section{My Information}
%
%       Here is my information.
%
% Otherwise the top of the section header will not line up with the top
% of the section. Of course, using a single comment character (%) on
% empty lines allows for the function of the first example with the
% readability of the second example.
\renewcommand{\section}[2]%
        {\pagebreak[2]\vspace{1.3\baselineskip}%
         \phantomsection\addcontentsline{toc}{section}{#1}%
         \hspace{0in}%
         \marginpar{
         \raggedright \scshape #1}#2}

% An itemize-style list with lots of space between items
\newenvironment{outerlist}[1][\enskip\textbullet]%
        {\begin{itemize}[#1]}{\end{itemize}%
         \vspace{-.6\baselineskip}}

% An environment IDENTICAL to outerlist that has better pre-list spacing
% when used as the first thing in a \section
\newenvironment{lonelist}[1][\enskip\textbullet]%
        {\vspace{-\baselineskip}\begin{list}{#1}{%
        \setlength{\partopsep}{0pt}%
        \setlength{\topsep}{0pt}}}
        {\end{list}\vspace{-.6\baselineskip}}

% An itemize-style list with little space between items
\newenvironment{innerlist}[1][\enskip\textbullet]%
        {\begin{compactitem}[#1]}{\end{compactitem}}

% To add some paragraph space between lines.
% This also tells LaTeX to preferably break a page on one of these gaps
% if there is a needed pagebreak nearby.
\newcommand{\blankline}{\quad\pagebreak[2]}

%--- Ignore what follows
\newcommand{\Ignore}[1]{}

% 

%%%%%%%%%%%%%%%%%%%%%%%% End Helper Commands %%%%%%%%%%%%%%%%%%%%%%%%%%%

%%%%%%%%%%%%%%%%%%%%%%%%% Begin CV Document %%%%%%%%%%%%%%%%%%%%%%%%%%%%

\begin{document}
\makeheading{Llohann Dallagnol Speranca}

\section{Address}
%
% NOTE: Mind where the & separators and \\ breaks are in the following
%       table.
%
% ALSO: \rcollength is the width of the right column of the table
%       (adjust it to your liking; default is 1.85in).
%

%--- Move two lines up then add contact info; this is a bit of a hack; also might want to move up a bit more
\vspace{-2em}
\begin{tabbing}
 Federal University of São Paulo xxxxxxxxxxxxxxl \= \kill
%\href{http://www.ime.unicamp.br/}
{Federal University of São Paulo}   \\
%\href{http://www.unicamp.br/unicamp/}
{Department of Science and Technology}
                           \> \textit{Cell:} +55 12 9 9788 7860 \\
Av. Cesare M. G. Lattes, 1201          \> \textit{E-mail:} {lsperanca@gmail.com, speranca@unifesp.br} \\
São José dos Campos  – São Paulo          \>  \textit{Nationality:} Brazilian, Italian \\
12247-014, Brazil 
\end{tabbing}

		\rule{\columnwidth}{.5pt}%


%\section{Citzenship}
%%
%Brazilian

\vspace{-1em}
\section{Research Interests}
%
My main interest is in Riemannian geometry, specially in its interplay between topology and curvature of manifolds with symmetry. Recently I have been implementing analytical methods and became interested for them. These includes elliptic PDEs, variational methods and geometric flows. I have also been working in Physics, Mathematical Physics and Combinatorics.

		\rule{\columnwidth}{.5pt}%

\vspace{-1em}
\section{Education}
%
\href{http://ime.unicamp.br/}{\textbf{State University of Campinas}},
Campinas, Brazil

%\blankline
\begin{outerlist}
	
\item[] Ph.D.,
        \href{http://www.ime.unicamp.br/}
             {Mathematics}
            May 2012
        \begin{innerlist}
        \item Thesis Title: \emph{Geometry and Topology of Cobordisms}
        \item Advisor:
              \href{http://lattes.cnpq.br/2287909020835559}
                   {Professor Carlos Durán} and Professor Alcibiades Rigas 
        \item Area of Study: Differential Topology, Riemannian Geometry
        \end{innerlist}

%\blankline

\item[] M.S.,
        \href{http://www.ime.unicamp.br/}
             {Mathematics}, August 2009
        \begin{innerlist}
	\item Dissertation Title: \emph{Exotic Phenomena in Geometry and Topology}
	\item Advisor:
	\href{http://lattes.cnpq.br/2287909020835559}
	{Professor Carlos Durán} 
	\item Area of Study: Differential Topology, Riemannian Geometry
\end{innerlist}

%\blankline

\item[] B.S.,
        \href{http://www.ime.unicamp.br/}
             {Mathematics}, December 2007
\end{outerlist}

		\rule{\columnwidth}{.5pt}%

\vspace{-1em}
\section{Positions}
%	\item[]
	 Assistent professor  at the {Federal University of São Paulo},  \hfill Since Spring 2017 \\São José dos Campos, São Paulo, Brazil
	
	\blankline
	
	Assistent professor  at the {Federal University of Paraná}, \hfill Spring 2014 to Spring 2017\\ Curitiba, Paraná, Brazil

%\blankline
%
%Postdoctoral fellow at {State University of Campinas} \hfill Spring 2013 to Spring 2014\\ Campinas, São Paulo, Brazil

\rule{\columnwidth}{.5pt}%

\vspace{-1em}
\section{Shor-term positions} \vspace{-1em}
\begin{innerlist}
	\item[2018-2019] Visiting Professor at the University of K\"oln, Germany
	\item[2012-2013] Postdoctoral Researcher for the State of São Paulo at the State University of Campinas
\end{innerlist}
\rule{\columnwidth}{.5pt}%

\vspace{-1em}
\section{Honors and Awards}\vspace{-1em}

Capes Thesis Award 2021 and Gutierrez Thesis Award 2021 as the supervisor of the thesis of Leonardo Cavenaghi

\blankline

Postdoctoral, PhD, Masters and Undergraduate grants, Foundation for Research Support of the State of São Paulo (FAPESP), 2008-2013


\blankline

Bronze Medal for the work `Geometric Aspects in Gauge Theory' -- III Journey  on Scientific Initiation, IMPA, Rio de Janeiro, 2006


\blankline

 Support for the event `Advanced Program in Geometry' in the Federal University of Paraná, CAPES 7197/2014-90, CAPES 7891/2014-15, CNPq 466425/2014-7,  PCDP  000076/15, \$4,000 \hfill Summer\,\,2015

\blankline

CNPq Universal Call, project `Riemannian Foliations under positive/nonnegative seccional curvature', 404266/2016-9 , \$4,000, \hfill 2016-2020

\blankline

 FAPESP Research Grant for the project `Geometry and topology under positive/nonnegative sectional curvature', 2017/10892-7, \$3,100, \hfill 2017-2018

\blankline

FAPESP Research Abroad Grant, `Classification and global properties of Riemannian foliations', 2017/19657-0, \$42,000, \hfill 2018-2019



\rule{\columnwidth}{.5pt}%


\section{Publications}  \vspace{-1.8em}
\begin{enumerate}
	\item Silva, R. M., Sperança, L. D. On the completeness of dual foliations on nonnegatively curved symmetric spaces, \textit{\href{https://arxiv.org/abs/2006.13809}{arxiv:2006.13809}} accepted Revista Matemática Iberoamericana.
	
	
	
	\item Cavenaghi, L. F., Sperança, L. D. A metric deformation on fiber bundles and applications, \textit{\href{https://arxiv.org/abs/1801.06576}{arxiv:1801.06576}}, accepted Advances in Geometry.
	
	
	\item  Cavenaghi, L. F., Sperança, L. D. On the Geometry of Some Equivariantly Related Manifolds,  
	\textit{ \href{https://academic.oup.com/imrn/advance-article-abstract/doi/10.1093/imrn/rny268/5194089}{International Mathematics Research
			Notices, no. 23, 9730–9768, 2020}}.
	%	\vspace{.1 in}
	
	\item Sperança, L. D., Weil, S.  The metric foliations on Euclidean spaces. Math. Z. 295, 1295–1299 2020.
	
	
	
	\item Sperança, L. D. On Riemannian foliations over positively curved manifolds. \textit{The Journal of Geometric Analysis}, v. 28(3), p. 2206-2224, 2017.
	%	\vspace{.1 in}  
	
	
	\item Sperança, L. D. Pulling back the Gromoll-Meyer construction and models of exotic spheres. \textit{Proceedings of the American Mathematical Society}, v. 144, p. 3181-3196, 2016.
	%    \vspace{.1 in}
	
	
	\item Durán, C. E., Sperança, L. D. Rigidity of flat sections on non-negatively curved pullback submersions, \textit{Manuscripta Mathematica}, v. 147, p. 511-525, 2015.
	%    \vspace{.1 in}
	
	
	\item Sperança, L. D. An identification of the Dirac operator with the parity operator. \textit{International Journal of Modern Physics D},  v. 97, p. 495-497, 2011.
	%    \vspace{.1 in}
	
	
	
	\item Sperança L. D. A note on the degree of symmetry of exotic spheres. \textit{Archiv der Mathematik (Printed ed.)},  v. 97, p. 495-497, 2011.
	%    \vspace{.1 in}  	  
	
	
	\item Durán, C. E. , Rigas, A., Sperança, L. D.  Bootstrapping Ad-equivariant maps, Diffeomorphims and Involutions. \textit{Matematica Contemporanea},  v. 35, p. 27-39, 2008.
	%    \vspace{.1 in}
	
	\item Jardim, M. B., Sperança, L. D. Nonsingular Complex Instantons on Euclidean Spacetime. \textit{International Journal of Geometric Methods in Modern Physics}, v. 05, p. 963, 2008.
	%    \vspace{.1 in}      
	
\end{enumerate}
\section{Preprints} \vspace{-1.8em}
\begin{enumerate}

\item Cavenaghi, L. F., do Ó, J., Sperança, L. D., The symmetric Kazdan--Warner problem and applications,\textit{\href{https://arxiv.org/abs/2106.14709}{arxiv:2106.14709}}, submitted to Acta Mathematica.


\item Sperança, L. D. Totally Geodesic Riemannian Foliations on Compact Lie Groups, \textit{\href{https://arxiv.org/abs/1703.09577}{arxiv:1703.09577}}, submitted to The American Journal of Mathematics.
%	\vspace{.1 in}


\item Cavenaghi, L. F., Silva, R. M., Sperança, L. D. Positive Ricci curvature through Cheeger deformation,
\textit{\href{https://arxiv.org/abs/1810.09725}{arxiv:1810.09725}}, submitted The Journal of Geometric Analysis (under review, with a positive referee report).
%	\vspace{.1 in}

\item  Mauro, P., Seco, L., Sperança, L. D., On the embeddability of the homogeneous Ricci flow and its collapses. Submitted to Annals of Global Annalisys and Geometry


\item Grama, L., Martins, R. M., Patrão, M., Seco, L., Sperança, L. D.
Global dynamics of the Ricci flow on flag manifolds with three isotropy summands, \textit{\href{https://arxiv.org/abs/2004.01511}{arxiv:2004.01511}}, submitted to Monatshefte f\"ur Mathematik.


\item Blinovsky, V. M., Sperança, L. D., The solution of the complete nontrivial cycle intersection problem for permutations 
\textit{\href{	https://arxiv.org/abs/1303.0053}{arxiv:1303.0053}}, submitted to Annals of the Brazilian Mathematical Society.


\item Sperança, L. D. An intrinsic curvature condition for submersions over Riemannian manifolds
\textit{\href{https://arxiv.org/abs/1706.09211}{arxiv:1706.09211}}.

\item Cavenaghi, L. F., Sperança, L. D. The curvature of convex sum of metrics and applications, \textit{\href{https://arxiv.org/abs/2106.14781}{arxiv:2106.14781}}.

%	\vspace{.1 in}






\end{enumerate}


%\rule{\columnwidth}{.5pt}%


\section{In preparation} \vspace{-1.8em}
\begin{enumerate}
\item Cavenaghi, L., Sperança, L. D., On the space of positive scalar curvature metrics with symmetries


\item Melo, M. M., Sperança, L. D., Metric deformations via grupoids and collapses

\item  Silva, R. M., Sperança, L. D., On the genericity  of  dual foliations with a single leaf


\end{enumerate}

%\rule{\columnwidth}{.5pt}%


\section{Book} \vspace{-1.8em}
\begin{enumerate}
	\item  Sperança, L. D. Grupos de Lie via exemplos: Topologia, Geometria e Física. ed. Rio de Janeiro SBM 2016 v.1 61p.
\end{enumerate}



\rule{\columnwidth}{.5pt}%
\vspace{-1em}

\section{Invited Talks}
%
\textbf{ On the space of PSC metrics with non-abelian symmetry }
\begin{innerlist}
	\item[]- Augsburg-Friburg-Kahrlsruhe A-Fri-Ka Riemannian Topology Seminar (online), 2021
\end{innerlist}

\textbf{ The projected homogeneous Ricci flow and its Collapses }
\begin{innerlist}
	\item[] Symmetry and Shape, Santiago de Compostela (Spain), 2021
	\item[] Mathematics Colloquium,  University of Friburg (Switzerland), 2021
\end{innerlist}

\textbf{ Positive Ricci curvature through Cheeger deformation  }
\begin{innerlist}
\item[] XII Summer School, Dynamical Systems Section,  University of Brasília (Brazil), 2020
\end{innerlist}

\blankline

\textbf{ The metric foliations on Euclidean spaces }
\begin{innerlist}
	\item[]- SP Geometry Seminar, State University of Campinas (Brazil), 2020
\end{innerlist}




\blankline

\textbf{ On the geometry of some equivariantly related manifolds }
\begin{innerlist}
\item[]- Dynamical Systems Seminar, University of Brasília (Brazil), 2020
\item[]- Oberseminar Topologie, University of Fribourg (Switzerland), 2019
\item[]- Obeseminar Geometrie, University of Freiburg (Germany), 2019
\item[]- Geometry Seminar, KU Leuven (Belguim), 2019
\end{innerlist}



\blankline

\textbf{ Totally Geodesic Riemannian Foliations on Compact Lie Groups }
\begin{innerlist}
\item[]- AMS Sectional Meeting, University of California at Riverside (USA), 2017
\item[]- Workshop on Curvature and Global Shape, University of M\ unster (Germany), 2017
\item[]- Geometry Seminar, IMPA (Brazil), 2017
\item[]- Symposium on Lie Theory and Applications, State University of Maringá (Brazil), 2016
\item[]- Geometry Seminar, State University of Campinas (Brazil), 2015
\end{innerlist}


\blankline

\textbf{ Riemannian foliations on positively curved manifolds }
\begin{innerlist}
	\item[]- Dynamical Systems Seminar, University of Brasília (Brazil), 2021
\item[]- Geometry Seminar, University of K\ oln (Germany), 2019
\item[]- Oberseminar Geometrie, University of M\"unster (Germany), 2015
\item[]- Geometry Seminar, University of São Paulo (Brazil), 2015
\item[]- Geometry Seminar, University of Santa Catarina (Brazil), 2015
\end{innerlist}


\blankline

\textbf{ Lorentz Algebras, Space-time symmetries and Dark-Matter }
\begin{innerlist}
	\item[]- II School  and Workshop in Lie Theory, State University of Maringa (Brazil), 2013
\end{innerlist}


\blankline

\textbf{ Geometric Aspects of Gauge Theory }
\begin{innerlist}
	\item[]- III Journey on Scientific Initiation, IMPA (Brazil), 2006
\end{innerlist}



\rule{\columnwidth}{.5pt}%

\vspace{-1em}
    
  \section{Students} 
  \textbf{PhD Thesis}
  	\begin{innerlist}
  	\item[2018--2020] Leonardo Francisco Cavenaghi: On metric deformations and applications, University of São Paulo -- awarded as the best thesis in Mathematics defended in Brazil 2020
  	\item[2018--2022] Renato J. Moura e Silva: Results on Riemannian foliations, State University of Campinas 
  	\end{innerlist}
  
  \blankline
  \
  \hspace{-0.5cm}\textbf{Masters Dissertation}
  \begin{innerlist}
  	\item[2013--2014] Aline Zanardini: Dirac Structures and their homotopy classification, Federal University of Paraná 
  \end{innerlist}
  
  \blankline
  
  \textbf{Undergraduate Students}
  \begin{innerlist}
  	\item[2017-2017]
  	Hilário Fernandes de Araújo Júnior: Homology and Homotopy in Geometric Topology, Federal University of São Paulo
  	\item[2016-2016] Felipe Ogima: Introduction to Ordinary Differential Equations, Federal University of Paraná
  	\item[2016-2016] Luciano Luzzi Junior: Introduction to Fiber Bundles, Federal University of Paraná
  	\item[2016-2016] Gabriel José Goulart Cardoso: Topology and Geometry in Geometric Quantization, Federal University of Paraná
  \end{innerlist}


\rule{\columnwidth}{.5pt}%

\vspace{-1em}
\section{Teaching Experience }
\textbf{\href{http://ime.unicamp.br}{State University of Campinas}}
\hfill {Summer 2013}
\begin{innerlist}
	\item[]- Advanced Linear Algebra (graduation)\hfill  Summer 2013
\end{innerlist}

\blankline

\textbf{\href{http://www.mat.ufpr.br/}{Federal University of Paraná}}
\hfill { 2014/02 to 2017/02}
\begin{innerlist}
	\item[]- Calculus \hfill Spring 2014, Fall 2015 
	\item[]- Calculus in multiple variables \hfill Spring 2014, Spring 2016, Fall 2016 
	\item[]- Linear Algebra \hfill Spring 2015, Fall 2015, Spring 2016 
	\item[]- Differential Geometry (Math major) \hfill Spring 2014
	\item[]- Algebraic Topology (Math major) \hfill Fall 2014
	\item[]- Topics in Geometry/Topology (Math Major) \hfill  Spring 2016
	\item[]- Advanced Linear Algebra (graduation) \hfill Spring 2015 
	\item[]- Ordinary Differential Equations (graduation) \hfill  Fall 2016 		
	\item[]- Geometry and Topology (PhD course on manifolds) \hfill Fall 2014
	
\end{innerlist}

\blankline

\textbf{\bfseries\href{https://www.unifesp.br/campus/sjc/}{Federal University of São Paulo}}
\hfill {2017/02 to Present}
\begin{innerlist}
	\item[]- Real Analysis II (Math Major) \hfill Fall 2021 (online)
	\item[]- Topics in Geometry: Optimal Transport \hfill Fall 2021 (online)
	\item[]- Ordinary Differential Equations \hfill Fall 2019, Fall 2020 (online)
	\item[]- Calculus\hfill  Spring 2018; Spring 2020, 2021 (online)
	\item[]- Multiple Variables Calculus\hfill  Spring 2019
	\item[]- Analysis on $\mathbb{R}^n$ II (graduation) \hfill Spring 2019 \item[]- Analysis on $\mathbb R^n$ (graduation) \hfill Fall 2019
	\item[]- Advanced Linear Algebra (graduation)\hfill Summer 2018 
	\item[]- Metric Spaces (Math major) \hfill Spring 2017 
	\item[]- Ordinary Differential Equations (Math major) \hfill Fall 2017
	\item[]- General Topology (graduation) \hfill Fall 2017 
	%	\item[]- Differential Geometry (Math major): Spring 2014
	%	\item[]- Algebraic Topology (Math major):Fall 2014
	%	\item[]- Geometry and Topology (PhD course on manifolds): Fall 2014
	%	\item[]- Linear Algebra: Spring 2015 (Mechanical Engineering), Fall 2015 (Production Engineering), Spring 2016 (Chemical Engineering)
	%	\item[]- Topics in Geometry/Topology (Introduction to Riemannian Geometry): Spring 2016
\end{innerlist}

\rule{\columnwidth}{.5pt}%

\vspace{-1em}
\section{Academic and\\ Professional Service}
%\subsection{Conferences and Workshops} 
%   	
\textbf{Organization of events}
   \begin{innerlist}
   \item[]- Advanced Summer Program in Geometry, Federal University of Paraná, 2015
   \item[]- III Geometry Symposium, Federal University of Paraná, 2015
   \item[]- V Geometry Symposium, Federal University of Paraná, 2017
   \end{innerlist}
   
   \blankline
   
   \textbf{Reviewer Service}
   \begin{innerlist}
\item[]- Project Reviewer for The São Paulo Research Foundation (Fapesp)
\item[]- Project Reviewer for National Counsil of Research (CNPq)
   \end{innerlist}
	 
	 \blankline
	 
	 \textbf{Referee Service}
		\begin{innerlist}
   \item[]- Advances in Applied Clifford Algebras
\item[]- International Journal of Geometric Methods in Modern Physics
\item[]- SIGMA
\item[]- Mediterranean Journal of Mathematics
		\end{innerlist}
				

%\end{outerlist}




\end{document}

%%%%%%%%%%%%%%%%%%%%%%%%%% End CV Document %%%%%%%%%%%%%%%%%%%%%%%%%%%%%
